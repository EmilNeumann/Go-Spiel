%
% API Documentation for API Documentation
% Module Main.main
%
% Generated by epydoc 3.0.1
% [Tue Jan 05 17:30:22 2016]
%

%%%%%%%%%%%%%%%%%%%%%%%%%%%%%%%%%%%%%%%%%%%%%%%%%%%%%%%%%%%%%%%%%%%%%%%%%%%
%%                          Module Description                           %%
%%%%%%%%%%%%%%%%%%%%%%%%%%%%%%%%%%%%%%%%%%%%%%%%%%%%%%%%%%%%%%%%%%%%%%%%%%%

    \index{Main \textit{(package)}!Main.main \textit{(module)}|(}
\section{Module Main.main}

    \label{Main:main}

%%%%%%%%%%%%%%%%%%%%%%%%%%%%%%%%%%%%%%%%%%%%%%%%%%%%%%%%%%%%%%%%%%%%%%%%%%%
%%                               Functions                               %%
%%%%%%%%%%%%%%%%%%%%%%%%%%%%%%%%%%%%%%%%%%%%%%%%%%%%%%%%%%%%%%%%%%%%%%%%%%%

  \subsection{Functions}

    \label{Main:main:main}
    \index{Main \textit{(package)}!Main.main \textit{(module)}!Main.main.main \textit{(function)}}

    \vspace{0.5ex}

\hspace{.8\funcindent}\begin{boxedminipage}{\funcwidth}

    \raggedright \textbf{main}()

    \vspace{-1.5ex}

    \rule{\textwidth}{0.5\fboxrule}
\setlength{\parskip}{2ex}
    Methode, die das Programm laufen lässt

\setlength{\parskip}{1ex}
    \end{boxedminipage}


%%%%%%%%%%%%%%%%%%%%%%%%%%%%%%%%%%%%%%%%%%%%%%%%%%%%%%%%%%%%%%%%%%%%%%%%%%%
%%                               Variables                               %%
%%%%%%%%%%%%%%%%%%%%%%%%%%%%%%%%%%%%%%%%%%%%%%%%%%%%%%%%%%%%%%%%%%%%%%%%%%%

  \subsection{Variables}

    \vspace{-1cm}
\hspace{\varindent}\begin{longtable}{|p{\varnamewidth}|p{\vardescrwidth}|l}
\cline{1-2}
\cline{1-2} \centering \textbf{Name} & \centering \textbf{Description}& \\
\cline{1-2}
\endhead\cline{1-2}\multicolumn{3}{r}{\small\textit{continued on next page}}\\\endfoot\cline{1-2}
\endlastfoot\raggedright \_\-\_\-p\-a\-c\-k\-a\-g\-e\-\_\-\_\- & \raggedright \textbf{Value:} 
{\tt \texttt{'}\texttt{Main}\texttt{'}}&\\
\cline{1-2}
\end{longtable}


%%%%%%%%%%%%%%%%%%%%%%%%%%%%%%%%%%%%%%%%%%%%%%%%%%%%%%%%%%%%%%%%%%%%%%%%%%%
%%                           Class Description                           %%
%%%%%%%%%%%%%%%%%%%%%%%%%%%%%%%%%%%%%%%%%%%%%%%%%%%%%%%%%%%%%%%%%%%%%%%%%%%

    \index{Main \textit{(package)}!Main.main \textit{(module)}!Main.main.mainGUI \textit{(class)}|(}
\subsection{Class mainGUI}

    \label{Main:main:mainGUI}
Definition des Fensters, in dem das Spiel ablaufen soll


%%%%%%%%%%%%%%%%%%%%%%%%%%%%%%%%%%%%%%%%%%%%%%%%%%%%%%%%%%%%%%%%%%%%%%%%%%%
%%                                Methods                                %%
%%%%%%%%%%%%%%%%%%%%%%%%%%%%%%%%%%%%%%%%%%%%%%%%%%%%%%%%%%%%%%%%%%%%%%%%%%%

  \subsubsection{Methods}

    \label{Main:main:mainGUI:__init__}
    \index{Main \textit{(package)}!Main.main \textit{(module)}!Main.main.mainGUI \textit{(class)}!Main.main.mainGUI.\_\_init\_\_ \textit{(method)}}

    \vspace{0.5ex}

\hspace{.8\funcindent}\begin{boxedminipage}{\funcwidth}

    \raggedright \textbf{\_\_init\_\_}(\textit{self}, \textit{master})

    \vspace{-1.5ex}

    \rule{\textwidth}{0.5\fboxrule}
\setlength{\parskip}{2ex}
    Klassenkonstruktor, wird aufgerufen, wenn ein Objekt dieser Klasse 
    instanziert wird

\setlength{\parskip}{1ex}
    \end{boxedminipage}

    \label{Main:main:mainGUI:initializeComponents}
    \index{Main \textit{(package)}!Main.main \textit{(module)}!Main.main.mainGUI \textit{(class)}!Main.main.mainGUI.initializeComponents \textit{(method)}}

    \vspace{0.5ex}

\hspace{.8\funcindent}\begin{boxedminipage}{\funcwidth}

    \raggedright \textbf{initializeComponents}(\textit{self})

    \vspace{-1.5ex}

    \rule{\textwidth}{0.5\fboxrule}
\setlength{\parskip}{2ex}
    Hier werden alle Komponenten des Fensters erzeugt

\setlength{\parskip}{1ex}
    \end{boxedminipage}

    \label{Main:main:mainGUI:setStone}
    \index{Main \textit{(package)}!Main.main \textit{(module)}!Main.main.mainGUI \textit{(class)}!Main.main.mainGUI.setStone \textit{(method)}}

    \vspace{0.5ex}

\hspace{.8\funcindent}\begin{boxedminipage}{\funcwidth}

    \raggedright \textbf{setStone}(\textit{self}, \textit{x}, \textit{y})

    \vspace{-1.5ex}

    \rule{\textwidth}{0.5\fboxrule}
\setlength{\parskip}{2ex}
    Methode zum setzen der Steine

\setlength{\parskip}{1ex}
      \textbf{Parameters}
      \vspace{-1ex}

      \begin{quote}
        \begin{Ventry}{x}

          \item[x]

            {\it (type=Integer {\textgreater}= 0 und {\textless}= 19)}

        \end{Ventry}

      \end{quote}

    \end{boxedminipage}

    \label{Main:main:mainGUI:setBack}
    \index{Main \textit{(package)}!Main.main \textit{(module)}!Main.main.mainGUI \textit{(class)}!Main.main.mainGUI.setBack \textit{(method)}}

    \vspace{0.5ex}

\hspace{.8\funcindent}\begin{boxedminipage}{\funcwidth}

    \raggedright \textbf{setBack}(\textit{self}, \textit{x}, \textit{y})

    \vspace{-1.5ex}

    \rule{\textwidth}{0.5\fboxrule}
\setlength{\parskip}{2ex}
    mit dieser Methode kann man Steine zurücknehmen, wenn man sie 
    versehentlich gesetzt hat

\setlength{\parskip}{1ex}
    \end{boxedminipage}

    \label{Main:main:mainGUI:draw}
    \index{Main \textit{(package)}!Main.main \textit{(module)}!Main.main.mainGUI \textit{(class)}!Main.main.mainGUI.draw \textit{(method)}}

    \vspace{0.5ex}

\hspace{.8\funcindent}\begin{boxedminipage}{\funcwidth}

    \raggedright \textbf{draw}(\textit{self})

    \vspace{-1.5ex}

    \rule{\textwidth}{0.5\fboxrule}
\setlength{\parskip}{2ex}
    Methode zum Zeichnen des Spielfelds mit den Steinen

\setlength{\parskip}{1ex}
    \end{boxedminipage}

    \label{Main:main:mainGUI:zug_speichern}
    \index{Main \textit{(package)}!Main.main \textit{(module)}!Main.main.mainGUI \textit{(class)}!Main.main.mainGUI.zug\_speichern \textit{(method)}}

    \vspace{0.5ex}

\hspace{.8\funcindent}\begin{boxedminipage}{\funcwidth}

    \raggedright \textbf{zug\_speichern}(\textit{self})

    \vspace{-1.5ex}

    \rule{\textwidth}{0.5\fboxrule}
\setlength{\parskip}{2ex}
    Methode zum speichern der Daten:

    \begin{itemize}
    \setlength{\parskip}{0.6ex}
      \item ob ein Stein gesetzt wurde; wenn ja,

      \item wo der Stein gesetzt wurde (Koordinaten) und ggf.

      \item welche Farbe der Stein hat

    \end{itemize}

\setlength{\parskip}{1ex}
    \end{boxedminipage}

    \label{Main:main:mainGUI:left_click}
    \index{Main \textit{(package)}!Main.main \textit{(module)}!Main.main.mainGUI \textit{(class)}!Main.main.mainGUI.left\_click \textit{(method)}}

    \vspace{0.5ex}

\hspace{.8\funcindent}\begin{boxedminipage}{\funcwidth}

    \raggedright \textbf{left\_click}(\textit{self}, \textit{event})

    \vspace{-1.5ex}

    \rule{\textwidth}{0.5\fboxrule}
\setlength{\parskip}{2ex}
    Methode, die aufgerufen wird, wenn mit der linken Maustaste geklickt 
    wird

\setlength{\parskip}{1ex}
    \end{boxedminipage}

    \label{Main:main:mainGUI:del_stone}
    \index{Main \textit{(package)}!Main.main \textit{(module)}!Main.main.mainGUI \textit{(class)}!Main.main.mainGUI.del\_stone \textit{(method)}}

    \vspace{0.5ex}

\hspace{.8\funcindent}\begin{boxedminipage}{\funcwidth}

    \raggedright \textbf{del\_stone}(\textit{self}, \textit{x}, \textit{y})

    \vspace{-1.5ex}

    \rule{\textwidth}{0.5\fboxrule}
\setlength{\parskip}{2ex}
    Methode zum schlagen von Steinen

\setlength{\parskip}{1ex}
    \end{boxedminipage}

    \label{Main:main:mainGUI:right_click}
    \index{Main \textit{(package)}!Main.main \textit{(module)}!Main.main.mainGUI \textit{(class)}!Main.main.mainGUI.right\_click \textit{(method)}}

    \vspace{0.5ex}

\hspace{.8\funcindent}\begin{boxedminipage}{\funcwidth}

    \raggedright \textbf{right\_click}(\textit{self}, \textit{event})

    \vspace{-1.5ex}

    \rule{\textwidth}{0.5\fboxrule}
\setlength{\parskip}{2ex}
    Methode, die aufgerufen wird, wenn mit der rechten Maustaste geklickt 
    wird

\setlength{\parskip}{1ex}
    \end{boxedminipage}

    \label{Main:main:mainGUI:double_click}
    \index{Main \textit{(package)}!Main.main \textit{(module)}!Main.main.mainGUI \textit{(class)}!Main.main.mainGUI.double\_click \textit{(method)}}

    \vspace{0.5ex}

\hspace{.8\funcindent}\begin{boxedminipage}{\funcwidth}

    \raggedright \textbf{double\_click}(\textit{self}, \textit{event})

    \vspace{-1.5ex}

    \rule{\textwidth}{0.5\fboxrule}
\setlength{\parskip}{2ex}
    Methode, die aufgerufen wird, wenn doppelt mit der linken Maustaste 
    geklickt wird

\setlength{\parskip}{1ex}
    \end{boxedminipage}

    \label{Main:main:mainGUI:fertigButton_click}
    \index{Main \textit{(package)}!Main.main \textit{(module)}!Main.main.mainGUI \textit{(class)}!Main.main.mainGUI.fertigButton\_click \textit{(method)}}

    \vspace{0.5ex}

\hspace{.8\funcindent}\begin{boxedminipage}{\funcwidth}

    \raggedright \textbf{fertigButton\_click}(\textit{self})

    \vspace{-1.5ex}

    \rule{\textwidth}{0.5\fboxrule}
\setlength{\parskip}{2ex}
    Methode, die aufgerufen wird, wenn auf den Fertig-Button geklickt wird.

\setlength{\parskip}{1ex}
    \end{boxedminipage}

    \label{Main:main:mainGUI:mouse_move}
    \index{Main \textit{(package)}!Main.main \textit{(module)}!Main.main.mainGUI \textit{(class)}!Main.main.mainGUI.mouse\_move \textit{(method)}}

    \vspace{0.5ex}

\hspace{.8\funcindent}\begin{boxedminipage}{\funcwidth}

    \raggedright \textbf{mouse\_move}(\textit{self}, \textit{event})

    \vspace{-1.5ex}

    \rule{\textwidth}{0.5\fboxrule}
\setlength{\parskip}{2ex}
    Methode, die aufgerufen wird, wenn mit der Maus über das Spielbrett 
    gefahren wird

\setlength{\parskip}{1ex}
    \end{boxedminipage}

    \label{Main:main:mainGUI:press_enter}
    \index{Main \textit{(package)}!Main.main \textit{(module)}!Main.main.mainGUI \textit{(class)}!Main.main.mainGUI.press\_enter \textit{(method)}}

    \vspace{0.5ex}

\hspace{.8\funcindent}\begin{boxedminipage}{\funcwidth}

    \raggedright \textbf{press\_enter}(\textit{self}, \textit{event})

    \vspace{-1.5ex}

    \rule{\textwidth}{0.5\fboxrule}
\setlength{\parskip}{2ex}
    Methode, die aufgerufen werden sollte, wenn die Eingabetaste gedrückt 
    wird

\setlength{\parskip}{1ex}
    \end{boxedminipage}


%%%%%%%%%%%%%%%%%%%%%%%%%%%%%%%%%%%%%%%%%%%%%%%%%%%%%%%%%%%%%%%%%%%%%%%%%%%
%%                          Instance Variables                           %%
%%%%%%%%%%%%%%%%%%%%%%%%%%%%%%%%%%%%%%%%%%%%%%%%%%%%%%%%%%%%%%%%%%%%%%%%%%%

  \subsubsection{Instance Variables}

    \vspace{-1cm}
\hspace{\varindent}\begin{longtable}{|p{\varnamewidth}|p{\vardescrwidth}|l}
\cline{1-2}
\cline{1-2} \centering \textbf{Name} & \centering \textbf{Description}& \\
\cline{1-2}
\endhead\cline{1-2}\multicolumn{3}{r}{\small\textit{continued on next page}}\\\endfoot\cline{1-2}
\endlastfoot\raggedright f\-r\-a\-m\-e\- & Festlegung der Größe&\\
\cline{1-2}
\raggedright s\-t\-e\-i\-n\-g\-e\-s\-e\-t\-z\-t\- & Variable, die erforderlich ist, um nicht mehrere Steine in einem 
          Zug setzen zu können&\\
\cline{1-2}
\raggedright s\-t\-o\-n\-e\-c\-o\-l\-o\-r\- & legt fest, welche Farbe die gesetzten Steine haben&\\
\cline{1-2}
\raggedright x\- & x-Koordinate des zuletzt gesetzen Steins&\\
\cline{1-2}
\raggedright y\- & y-Koordinate des zuletzt gesetzten Steins&\\
\cline{1-2}
\raggedright z\-u\-g\-l\-i\-s\-t\-e\- & Spielverlauf in Form eines Text-Strings aus aneinandergereihten 
          Koordinaten&\\
\cline{1-2}
\raggedright d\-e\-l\-e\-t\-e\-d\-\_\-w\-h\-i\-t\-e\-\_\-s\-t\-o\-n\-e\-s\- & Anzahl der geschlagenen  weißen Steine&\\
\cline{1-2}
\raggedright d\-e\-l\-e\-t\-e\-d\-\_\-b\-l\-a\-c\-k\-\_\-s\-t\-o\-n\-e\-s\- & Anzahl der geschlagenen schwarzen Steine&\\
\cline{1-2}
\end{longtable}

    \index{Main \textit{(package)}!Main.main \textit{(module)}!Main.main.mainGUI \textit{(class)}|)}
    \index{Main \textit{(package)}!Main.main \textit{(module)}|)}
